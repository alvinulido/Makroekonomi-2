\documentclass[letter,11pt]{article}
%%%%%%%%%%%%%%%%%%%%%%%%%%%%%%%%%%%%%%%%%%%%%%%%
% 2. Packages
%%%%%%%%%%%%%%%%%%%%%%%%%%%%%%%%%%%%%%%%%%%%%%%%

\usepackage[top = 2cm, bottom = 2cm, left = 2cm, right = 2cm]{geometry} 

% The following two packages - multirow and booktabs - are needed to create nice looking tables.
\usepackage{IEEEtrantools}
\usepackage{setspace}
\usepackage{exercise}
\usepackage{multicol}
 \usepackage{nccmath}

\setlength{\parindent}{0in}

% Package to place figures where you want them.
\usepackage{float}

% The fancyhdr package let's us create nice headers.
\usepackage{fancyhdr}
\usepackage{amsmath}
\usepackage{amsfonts}
\usepackage{amssymb}
\DeclareMathOperator*{\argmax}{arg\,max}
\DeclareMathOperator*{\argmin}{arg\,min}
\let\otheriint\iint
\let\iint\relax
%%%%%%%%%%%%%%%%%%%%%%%%%%%%%%%%%%%%%%%%%%%%%%%%
% 3. Header (and Footer)
%%%%%%%%%%%%%%%%%%%%%%%%%%%%%%%%%%%%%%%%%%%%%%%%

\pagestyle{fancy} 
\fancyhf{}
\lhead{\footnotesize Macroeconomics 2: Homework 1}
\rhead{\footnotesize{\textit{Lumbanraja, Alvin Ulido}}} %<---- Fill in your lastnames.
\cfoot{\footnotesize \thepage} 

%%%%%%%%%%%%%%%%%%%%%%%%%%%%%%%%%%%%%%%%%%%%%%%%
% 4. Your document
%%%%%%%%%%%%%%%%%%%%%%%%%%%%%%%%%%%%%%%%%%%%%%%%
\begin{document}
\thispagestyle{empty} % This command disables the header on the first page. 
\begin{tabular}{p{17cm}} % This is a simple tabular environment to align your text nicely 
{\large \bf Homework 1} \\
Macroeconomics 2 \\ Beta Yulianita Gitaharie/Eugenia Mardanugraha  \\ {\bf{TA: Alvin Ulido Lumbanraja}}\\
\hline % \hline produces horizontal lines.
\\
\end{tabular} % Our tabular environment ends here.
\vspace*{0.3cm} % Now we want to add some vertical space in between the line and our title.
% Up until this point you only have to make minor changes for every week (Number of the homework). Your write up essentially starts here.

\begin{enumerate}

% 	Question 1
	%% 			1(a)
\item Prove that matrix multiplication is not commutative! (Hint: use online resources on how to write a proof. This is a good exercise for elementary real analysis)
\item Find the inverse of the following matrices. Provide explanation for your answer if necessary
\begin{multicols}{4}
\begin{enumerate}

\item $\begin{pmatrix}
2 & 1 
\\
1 & 2 
\\ 
3 & 5
\end{pmatrix}$ \\

\item $\begin{pmatrix}
1 & 0 & 0
\\
0 & 1 & 0
\\ 
0 & 0 & 1
\end{pmatrix}$ \\

\item $\begin{pmatrix}
\alpha & \beta & \gamma
\\
1 & 0 & 1
\\ 
0 & 1 & \delta
\end{pmatrix}$ \\

\item $\begin{pmatrix}
1 & 2 & 3
\\
1 & 0 & 1
\\ 
3 & 4 & 7
\end{pmatrix}$
\end{enumerate}
\end{multicols} 

Notes: 
\begin{itemize}
\item For each of the question above, you need to first determine if you can invert the matrix. If the matrix is non-invertible, explain why.
\item For question (c), determine if the matrix is invertible for all $\alpha, \beta, \gamma, \delta \in \mathbb{Z}$ (i.e. for all integer values). In either case, show your proof!
\item Before submitting your answer, in case you haven't noticed that technology exists, check your answer in Matlab or Wolfram Alpha first. This is 2020 for heaven's sake.
\end{itemize}


\item Solve the following national income model, using either matrix inversion or Cramer's rule
\begin{IEEEeqnarray}{rCl}
    Y &=& C + I + G  \nonumber \\
    C &=& \alpha + \beta(Y-T) \nonumber \\
    T &=& \tau_y Y \nonumber
\end{IEEEeqnarray}

(\textit{Hint: the only variables that are assumed to be exogenous in this model are $I$ and $G$})

\item Assume the following production function in the economy:
\begin{IEEEeqnarray}{rCl}
y &=& Ak^{\alpha} n^{\beta}  \nonumber
\end{IEEEeqnarray}
Where $k$ represents capital, $n$ represents labor, and $\alpha, \beta \in (0,1), \alpha+\beta=1$. 
\begin{enumerate}
\item If we assume a perfectly competitive economy in its steady state, find the competitive level of wage and rent in this economy! (remember the principle that wage and rent should be equal to their respective marginal products)
\item Explain what would happen to level of steady state wage and rent if we assume positive technological shock! (i.e. $\tilde{A} > A$)
\end{enumerate}

\item You should be familiar with Fisher's quantity theory of money from monetary economics course
\begin{IEEEeqnarray}{rCl}
MV = PY  \nonumber
\end{IEEEeqnarray}
Using total differentiation, prove that the rate of growth in money supply is proportional to inflation rate if velocity of money is constant (that is $\frac{dV}{V}=0$) and economic growth is constant (that is $g=\frac{dY}{Y}\in \mathbb{R}$). You do not need to worry about transmission lag, dynamic inconsistencies, or rational expectation for this question
\newline\newline
(\textit{Hint: you need to transform the equation into natural log form first before applying the differentiation, given the very neat property of differentiation for log})


\end{enumerate}

\clearpage
\end{document}