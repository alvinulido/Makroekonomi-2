\documentclass[letter,11pt]{article}
%%%%%%%%%%%%%%%%%%%%%%%%%%%%%%%%%%%%%%%%%%%%%%%%
% 2. Packages
%%%%%%%%%%%%%%%%%%%%%%%%%%%%%%%%%%%%%%%%%%%%%%%%

\usepackage[top = 2cm, bottom = 2cm, left = 2cm, right = 2cm]{geometry} 

% The following two packages - multirow and booktabs - are needed to create nice looking tables.
\usepackage{IEEEtrantools}
\usepackage{setspace}
\setlength{\parindent}{0in}

% Package to place figures where you want them.
\usepackage{float}

% The fancyhdr package let's us create nice headers.
\usepackage{fancyhdr}
\usepackage{array}
\usepackage{amsmath}
\usepackage{amssymb}
\usepackage[svgnames]{xcolor}
\usepackage{graphicx}
\usepackage{sectsty}
\usepackage[shortlabels]{enumitem}
\DeclareMathOperator*{\argmax}{arg\,max}
\DeclareMathOperator*{\argmin}{arg\,min}
\let\otheriint\iint
\let\iint\relax
%%%%%%%%%%%%%%%%%%%%%%%%%%%%%%%%%%%%%%%%%%%%%%%%
% 3. Header (and Footer)
%%%%%%%%%%%%%%%%%%%%%%%%%%%%%%%%%%%%%%%%%%%%%%%%

\pagestyle{fancy} 
\fancyhf{}
\lhead{\footnotesize Macroeconomics 2: Homework 3}
\rhead{\footnotesize{\textit{Lumbanraja, Alvin Ulido}}} %<---- Fill in your lastnames.
\cfoot{\footnotesize \thepage} 

%%%%%%%%%%%%%%%%%%%%%%%%%%%%%%%%%%%%%%%%%%%%%%%%
% 4. Your document
%%%%%%%%%%%%%%%%%%%%%%%%%%%%%%%%%%%%%%%%%%%%%%%%
\begin{document}
\newcolumntype{P}[1]{>{\centering\arraybackslash}p{#1}}
\thispagestyle{empty} % This command disables the header on the first page. 
\begin{tabular}{p{17cm}} % This is a simple tabular environment to align your text nicely 
{\large \bf Homework 3} \\
Macroeconomics 2 \\ Beta Yulianita Gitaharie/Eugenia Mardanugraha  \\ {\bf{TA: Alvin Ulido Lumbanraja}}\\
\hline % \hline produces horizontal lines.
\\
\end{tabular} % Our tabular environment ends here.
% Up until this point you only have to make minor changes for every week (Number of the homework). Your write up essentially starts here.

\begin{center}
\textbf{Due Date: October 22, 2020, 17.00 WIB \\ Late Penalty Applies}
\end{center}


Assume the following system of equations for our simple closed economy:
\begin{IEEEeqnarray}{rCl}
Y &=& C(Y) + I(Y,r) + G\\
L(Y,i) &=& \frac{M}{P} \\
H\left[\frac{(Y-\bar{Y})}{\bar{Y}}\right] &=& \frac{\dot{P}}{P} -\pi
\end{IEEEeqnarray}

\textbf{Questions:}
\begin{enumerate}

% 	Question 1
	%% 			1(a
\item Characterize the slope of \textit{IS} and \textit{LM}
\item What restrictions should we put in place in order for the slope of  \textit{IS} to be negative?
\item Explain mathematically why the convergence requirement dictates that the slope of \textit{LM} should be larger than the slope of \textit{IS}. (Hint: use the analysis of $\frac{d\dot{P}}{dP}$ from page 39-40)!
\item Explain intuitively why the convergence requirement dictates that the slope of \textit{LM} should be larger than the slope of \textit{IS}!
\item Assuming static expectation, derive the quantity theory of money from the  \textit{IS}/\textit{LM} relationship!
\end{enumerate}

\end{document}